\documentclass[%
%a5paper,                           % alle weiteren Papierformat einstellbar
%landscape,                     % Querformat
10pt,                               % Schriftgr§§e (12pt, 11pt (Standard))
%BCOR1cm,                           % Bindekorrektur, bspw. 1 cm
%DIVcalc,                           % f§hrt die Satzspiegelberechnung neu aus
%                                             s. scrguide 2.4
%twoside,                           % Doppelseiten
%twocolumn,                     % zweispaltiger Satz
halfparskip*,               % Absatzformatierung s. scrguide 3.1
%headsepline,                   % Trennline zum Seitenkopf  
%footsepline,                   % Trennline zum Seitenfu§
%titlepage,                     % Titelei auf eigener Seite
%normalheadings,            % §berschriften etwas kleiner (smallheadings)
%idxtotoc,                      % Index im Inhaltsverzeichnis
%liststotoc,                    % Abb.- und Tab.verzeichnis im Inhalt
%bibtotoc,                      % Literaturverzeichnis im Inhalt
%abstracton,                    % §berschrift §ber der Zusammenfassung an   
%leqno,                         % Nummerierung von Gleichungen links
%fleqn,                             % Ausgabe von Gleichungen linksb§ndig
%draft                              % §berlangen Zeilen in Ausgabe gekennzeichnet
%DIV = 14,
%monochrome,                        % schwarz wei§ output
]
{scrartcl}

\usepackage[
    papersize={1189mm,1682mm},
    landscape,
    top=0cm,
    bottom=0cm,
    left=0cm,
    right=0cm
]{geometry}



%% Deutsche Anpassungen %%%%%%%%%%%%%%%%%%%%%%%%%%%%%%%%%%%%%
\usepackage{amsmath,amssymb,mathtools,stmaryrd}
\usepackage[ngerman]{babel}
\usepackage[T1]{fontenc}
%\usepackage[ansinew]{inputenc}
\usepackage[utf8]{inputenc}

\usepackage{lmodern} %Type1-Schriftart f§r nicht-englische Texte

\usepackage[nomath]{kpfonts}
\renewcommand*\familydefault{\sfdefault}

\usepackage{tikz}
\usetikzlibrary{mindmap,trees, shadows, backgrounds, calc, decorations.text}



%% Math abbreviations %%%%%%%%%%%%%%%%%%%%%%%%%%%%%%%%%%%%%%%
\let\bb\mathbb
\let\cal\mathcal
\let\fr\mathfrak
\newcommand{\F}{\cal{F}}
\newcommand{\G}{\cal{G}}
\renewcommand{\O}{\cal{O}}
\renewcommand{\L}{\cal{L}}
\newcommand{\U}{\cal{U}}
\newcommand{\M}{\cal{M}}
\newcommand{\R}{\bb{R}}
\newcommand{\N}{\bb{N}}
\newcommand{\C}{\bb{C}}
\newcommand{\Q}{\bb{Q}}
\newcommand{\Z}{\bb{Z}}
\newcommand{\A}{\bb{A}}
\renewcommand{\P}{\bb{P}}
\newcommand{\rest}[1]{\big|_{#1}}
\newcommand{\inv}{^{-1}}
\newcommand{\fis}{^{\#}}
\newcommand{\kat}[1]{\mathbf{#1}}
\newcommand{\PSh}{\kat{PSh}}
\newcommand{\Sh}{\kat{Sh}}
\newcommand{\Ring}{\kat{Ring}}
\newcommand{\Ab}{\kat{Ab}}
\newcommand{\Top}{\kat{Top}}
\newcommand{\Sch}{\kat{Sch}}
\newcommand{\Moduln}[1]{#1\text{-}\kat{Mod}}
\newcommand{\affSch}{\kat{Sch}^\kat{aff}}
\newcommand{\op}{^\mathrm{op}}
\newcommand{\m}{\fr{m}}
\newcommand{\p}{\fr{p}}
\newcommand{\q}{\fr{q}}
\renewcommand{\a}{\fr{a}}
\renewcommand{\b}{\fr{b}}
\newcommand{\ideal}{\vartriangleleft}
\newcommand{\immersion}{\ \tikz[baseline=-0.6ex]{
    \draw[/tikz/commutative diagrams/immersion,right hook->]
        (0,0) -- +(1.3em,0);}\ }
\newcommand{\oimmersion}{\ \tikz[baseline=-0.6ex]{
    \draw[/tikz/commutative diagrams/offene immersion,right hook->]
        (0,0) -- +(1.3em,0);}\ }
\DeclareMathOperator{\Obj}{Obj}
\DeclareMathOperator{\Morph}{Morph}
\DeclareMathOperator{\Hom}{Hom}
\DeclareMathOperator{\Aut}{Aut}
\DeclareMathOperator{\im}{im}
\DeclareMathOperator{\Spec}{Spec}
\DeclareMathOperator{\Proj}{Proj}
\DeclareMathOperator{\Nil}{Nil}
\DeclareMathOperator{\Quot}{Quot}
\DeclareMathOperator{\charak}{char}
\DeclareMathOperator{\Span}{span}
\DeclareMathOperator{\res}{res}
\DeclareMathOperator{\pr}{pr}
\DeclareMathOperator{\id}{id}
\DeclareMathOperator{\Div}{Div}
\DeclareMathOperator{\Pic}{Pic}
\DeclareMathOperator{\supp}{supp}
\DeclareMathOperator{\codim}{codim}
\DeclareMathOperator{\mult}{mult}
\makeatletter
\let\div\@undefined
\makeatother
\DeclareMathOperator{\div}{div}
\DeclareMathOperator{\CaCl}{CaCl}
\newcommand{\funcdef}[1]{%
    \begin{array}[t]{>{\displaystyle}r>{\displaystyle}c>{\displaystyle}l}
    #1\end{array}}
\let\xto\xrightarrow

\newcommand{\cl}[1]{\overline{\{#1\}}}

\newcommand{\osubset}{\subseteq^\circ}



%Zus§tzliche Fontgr§§en
\newcommand{\HUGE}{\fontsize{30}{35}\selectfont}
\newcommand{\HUGEI}{\fontsize{40}{48}\selectfont}
\newcommand{\HUGEII}{\fontsize{60}{70}\selectfont}
\newcommand{\HUGEIII}{\fontsize{150}{100}\selectfont}


\colorlet{colDefinitionDraw}{green!80!black}
\colorlet{colDefinitionFill}{green!10}
\colorlet{colAnschauungDraw}{purple!80!black}
\colorlet{colAnschauungFill}{purple!10}
\colorlet{colSatzDraw}{blue!80!black}
\colorlet{colSatzFill}{blue!10}


%Functions
%paperwidth = 4785mm
%paperheight = 3383mm
\newcommand{\distElliptical}[1]{%
    \pgfmathparse{55*sqrt((\paperheight/\paperwidth*sin(#1))^2+cos(#1)^2)}\pgfmathresult cm}
  %  0.8*sqrt((round(\paperheight/10)*sin(#1))^2 + (round(\paperwidth/10)*cos(#1))^2)}\pgfmathresult mm}

\begin{document}

%% Lengths %%%%%%%%%%%%%%%%%%%%%%%%%%%%%%%%%%%%%%%%%%%%%%%%%%
\abovedisplayskip0ex plus0.5ex minus0.5ex
\belowdisplayskip0ex plus0.5ex minus0.5ex
\setlength{\abovedisplayshortskip}{0ex plus0.5ex minus0.5ex}
\setlength{\belowdisplayshortskip}{0ex plus0ex minus1ex}

\pagestyle{empty}
\begin{tikzpicture}[overlay]
\begin{scope}[shift={(0.5\paperwidth, -0.5\paperheight)}, 
    huge mindmap,
    root concept/.append style={
      circular drop shadow,
      concept color=black,
      fill=white, line width=5ex,
      text=black, font=\HUGEIII\bfseries\scshape,
      inner sep=0pt,
      text width=25cm},
    level 1 concept/.append style={
      circular drop shadow,
      concept color=black,
      fill=white, line width=2ex,
      text=black, font=\HUGEI\scshape,
      inner sep=0pt,
      text width=9cm},
    level 2 concept/.append style={
      circular drop shadow,
      concept color=black,
      fill=white, line width=1ex,
      text=black, font=\Large\scshape,
      inner sep=0pt,
      text width=3cm}]
    
    %styles
    \pgfkeys{tikz/definition concept/.style={
      circular drop shadow,
      concept color=colDefinitionDraw,
      fill=colDefinitionFill, draw=colDefinitionDraw,
      line width=2pt,
      inner sep=0pt,
      text width=3cm,
      normal},
      %
      tikz/satz concept/.style={
      circular drop shadow,
      concept color=colSatzDraw,
      fill=colSatzFill, draw=colSatzDraw,
      line width=2pt,
      inner sep=0pt,
      text width=4cm,
      normal}
    }
    %sizes
    \pgfkeys{tikz/normal/.style={font=\normalfont},
        tikz/important/.style={font=\large},
        tikz/very important/.style={font=\Large},
        tikz/very very important/.style={font=\Huge},
    }
      
    \node[concept, root concept] {Schema\\ theorie}
    % §1 Lokal geringte Räume
    child[grow=0, level distance=55cm]{ node[concept, level 1 concept] (section lokal geringte raume) {Lokal geringte Räume}
    }
    % §2 Affine Schemata
    child[grow=30, level distance=55cm]{ node[concept, level 1 concept] (section affine schemata) {Affine Schemata}
    }
    % §3 Beispiele
    child[grow=60, level distance=50cm]{ node[concept, level 1 concept] (section Beispiele) {Beispiele}
    }
    % §4 Projektive Schemata
    child[grow=90, level distance=45cm]{ node[concept, level 1 concept] (section projektive schemata) {Projektive Schemata}
    }
    % §5 Eigenschaften von Schemata
    child[grow=120, level distance=50cm]{ node[concept, level 1 concept] (section eigenschaften von schemata) {Eigen\-schaften von Schemata}
    }
    % §6 Faserprodukt
    child[grow=150, level distance=55cm]{ node[concept, level 1 concept] (section faserprodukt) {Faserprodukt}
    }
    % §7 glatt, regulär und normal
    child[grow=180, level distance=55cm]{ node[concept, level 1 concept] (section glatt regular normal) {Glatt, regulär und normal}
    }
    % §8 k-Varietäten
    child[grow=210, level distance=55cm]{ node[concept, level 1 concept] (section k-varietaten) {$k$-Varietäten}
    }
    % §9 Der Punktefunktor
    child[grow=240, level distance=50cm]{ node[concept, level 1 concept] (section punktefunktor) {Der Punktefunktor}
    }
    % §10 O_X Moduln
    child[grow=270, level distance=45cm]{ node[concept, level 1 concept] (section ox moduln) {$\O_X$-Moduln}
    }
    % §11 Divisoren
    child[grow=300, level distance=50cm]{ node[concept, level 1 concept] (section divisoren) {Divisoren}
        % §11.1 Cartier-Divisoren 
        child[grow=0, level distance=15cm] { node[concept, level 2 concept] (subsection cartier-divisoren) {Cartier-Divisoren}
            %
            [level 3 concept/.append style={clockwise from=0, level distance=8cm, sibling angle=50}]
            child[definition concept]{ node[concept, definition concept] (def garbe der meromorphen funktionen) 
                {Garbe der Keime meromorpher Funktionen}
            }
            child[definition concept]{ node[concept, definition concept] (def cartier-divisoren) 
                {Cartier-Divisoren\\$\Div(X)$}
                child[definition concept, level distance=5cm, grow=-90]{ node[concept, definition concept] (def cartier-divisoren-klassengruppe) 
                    {Cartier-Divisoren-Klassengruppe\\$\CaCl(X)$}
                }
            }
            child[satz concept]{ node[concept, satz concept] (lemma divisoren geben eindeutige untergarbe)
                    {Divisor $D = (U_i,f_i)_{i\in I} \in \Div(X)$ gibt eindeutige Untergarbe
                    \[\O_X(D) \subseteq \cal K_X\]
                    mit $\O_X(D)\rest{U_i} = f_i\inv \O_X\rest{U_i}$}
            }
            child[satz concept]{ node[concept, satz concept] (satz rho induziert additiven ghom)
                    {$\rho: \Div(X) \to\Pic(X),\ D \mapsto \O_X(D)$ induziert
                    additiven Gruppenhomomorphismus
                    \[ \rho: \CaCl(X) \hookrightarrow \Pic(X).\]
                    Es ist $\im(\rho) = \{[\L] \in \Pic(X) \mid \L \subseteq \cal K_X\}$.}
                child[satz concept, level distance=6cm, grow=-90]{ node[concept, satz concept] (sat x ganz dann rho iso)
                    {$X$ ganz. Dann $\rho:\CaCl \xto{\cong}\Pic(X)$.}
                }
            }
        }
        child[grow=50, level distance=15cm] { node[concept, level 2 concept] (subsection weil-divisoren) {Weil-Divisoren}
            %
            [level 3 concept/.append style={clockwise from=0, level distance=8cm, sibling angle=40}]
            child[definition concept]{ node[concept, definition concept] (def zykel)
                {Zykel\\ $Z = \sum_{x\in X} n_x \cl x$}
                child[definition concept, level distance=4cm]{ node[concept, definition concept] (def zykel von kodim 1)
                    {von Kodimension 1\\
                    $Z^1(X)$}
                }
            }
            child[definition concept]{ node[concept, definition concept] (def weil-divisoren)
                {Weil-Divisoren\\ $Z^1(X)$}
            }
            child[satz concept]{ node[concept, satz concept] (satz x noethersch integer f in o x eta dann f invbar in fast allen o x x)
                {$X$ noethersch, integer. $0\neq f \in K(X) = \O_{X,\eta}$. Dann
                $f \in \O_{X,x}^\times$ für f.a. $x$ mit $\dim\O_{X,x}=1$.}
            }
            child[definition concept]{ node[concept, definition concept] (def hauptdivisor zu f)
                {Hauptdivisor \\$(f)$}
            }
        }
    };
\end{scope}


\newcommand{\definition}[2][]{
    \node[#1, fill=colDefinitionFill, draw=colDefinitionDraw,
        fill opacity=0.5, draw opacity=0.5, text opacity=0.8]{#2};
}

\newcommand{\anschauung}[2][]{
    \node[#1, fill=colAnschauungFill, draw=colAnschauungDraw,
        fill opacity=0.5, draw opacity=0.5, text opacity=0.8]{#2};
}

%annotations
\begin{scope}
    \pgfkeys{/tikz/in concept/.style={at=(#1.north), yshift=-1cm)}}
    \pgfkeys{/tikz/annotation/.style={}}
    \node[in concept={section divisoren}, annotation] {\S 11};
    \node[in concept={subsection cartier-divisoren}, annotation] {\S 11.1};
\end{scope}

%environments
\begin{scope}
    % §11.1 Cartier-Divisoren
    \definition[at=($(def garbe der meromorphen funktionen) + (80:3cm)$), 
        text width=5cm]
        {\[\cal K_X := a (\cal K_X')\] mit
        \[ \cal K_X': U \mapsto \O_X(U)[S(U)\inv]\]}
    \anschauung[at=($(def garbe der meromorphen funktionen) + (160:4cm)$), 
        text width=3cm]
        {Ist $X$ ganz, so $\cal K_X= \underline{K(X)}$.}
    \definition[at=($(def cartier-divisoren) + (45:4cm)$), 
        text width=6cm]
        {\[\Div(X) := \Gamma(X,\cal K_X^\times/\O_X^\times)\]
        \emph{Hauptdivisoren} sind Bild von
        \[\div: \cal K_X^\times(X) \to \Div(X),\ f \mapsto \div(f)\]}
    \anschauung[at=($(def cartier-divisoren) + (-30:4.5cm)$), 
        text width=5cm]
        {\[D = [(U_i,f_i)_{i\in I}]\] mit 
        $X = \cup U_i$, $f_i = \tfrac{a_i}{b_i}$, $a_i,b_i \in \O_X^\times(U_i)$
        halmweise regulär.}
    \definition[at=($(def cartier-divisoren-klassengruppe) + (0:4.5cm)$), 
        text width=4cm]
        {\[\CaCl(X) := \Div(X)\big/\sim\]
        mit $\sim$ linearer Äquivalenz.}
    % §11.2 Weil-Divisoren
    \definition[at=($(def zykel) + (80:3cm)$), 
        text width=8cm]
        {Element von 
        \[\Z^{(X)} := \{Z = \sum_{x\in X} n_x \cl x \mid
            n_x \in \Z,\ n_x = 0\text{ f.a. }x\}\]}
    \definition[at=($(def zykel von kodim 1) + (-90:2.5cm)$), 
        text width=3cm]
        { $\codim_X \cl x = 1$ $\Leftrightarrow$
        $\dim \O_{X,x} = 1$}
    \definition[at=($(def weil-divisoren) + (0:4cm)$), 
        text width=4cm]
        {Falls $X$ noethersch und integer.}
    \anschauung[at=($(satz x noethersch integer f in o x eta dann f invbar in fast allen o x x) + (90:4cm)$), 
        text width=5cm]
        {Im affinen ist $f$ rationale Funktion. Der Nenner hat dann nur endlich viele Nullstellen
        und damit liegt $f$ in fast allen $\O_{X,x} = A_\p$, wo $x$ keine Nullstelle des Nenners ist.}
     \definition[at=($(def hauptdivisor zu f) + (130:3.5cm)$), 
        text width=7cm]
        {$X$ noethersch, normal. $f\in K(X) \setminus\{0\}$.
        \[(f) := \sum_{x\in X\atop \dim\O_{X,x} = 1} \mult_x(f) \cdot \cl x
        \quad \in Z^1(X)\]}
\end{scope}
\end{tikzpicture}

\end{document}
